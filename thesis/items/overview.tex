% \specialsection{Обзор существующих решений}
\section{Обзор}

\subsection{Предметная область}

    \paragraph{Kotlin.} \todo{}

    \paragraph{Kotlin Multiplatform\cite{kotlin-multiplatform}.} \todo{}

    \paragraph{Целевые платформы JS (JavaScript).} Компилятор Kotlin/JS\cite{kotlin-js} позволяет транслировать Kotlin код в JavaScript код, предназначенный для исполнения в двух средах (в основном): исполнение в браузере, когда коду доступны API веб-браузера и функциональность для управления содержимым веб-страницы, и исполнение на стороне сервера --- в данном случае доступна функциональность Node.js, которая, например, предоставляет доступ к файловой системе операционной системы\cite{nodejs-browser-diff}. Важно понимать отличие этих двух вариантов, поскольку иногда мультиплатформенные проекты заявляют поддержку JavaScript как платформы, но не уточняют, в какой именно среде они могут работать. Далее в тексте данные целевые платформы будет обозначаться как JS (browser) и JS (node.js) соответственно.

    \paragraph{Вирутальная файловая система.} \todo{}
%subsection

\subsection{Существующие решения}
    Для поиска существующих решений были сделаны следующие запросы в поисковую систему Google:
    \begin{itemize}
        \item <<kotlin multiplatform io>>
        \item <<kotlin multiplatform filesystem>>
        \item <<kotlin multiplatform file>>
        \item <<kotlin multiplatform storage>>
    \end{itemize}
    Аналогичные запросы были сделаны и на русском языке, но никаких дополнительных результатов это не принесло.

    Также был произведен поиск по релевантным теме вопросам на портале StackOverflow\cite{stackoverflow}. Несколько релевантных вопросов нашлось\cite{so-file-io-with-kotlin-multiplatform,so-read-write-file-in-kotlin-native-ios-side}, но дополнительных результатов они не принесли.
    Дополнительно был произведен поиск по репозиториям на портале GitHub\cite{github}. Кроме решений, перечисленных ниже, был обнаружен курируемый сообществом репозиторий со списком Kotlin Multiplatform библиотек\cite{gh-kotlin-multiplatform-libs}.

    Список полученных релевантных решений можно разделить на три категории.

    \newcommand{\existingsolution}[3]{\item \textbf{#1}#2\par#3}

    \subsubsection{Решения, предоставляющие интерфейс виртуальных файловых систем}\label{existing-vfs-solutions}
        \begin{itemize}
            \existingsolution{kile}{\cite{gh-kile}}{
                Последнее изменение кода --- июль 2020 г., сайт с документацией недоступен.
                Из кода ясно, что (возможно) поддерживаются платформы JVM, JS (Node.js).
                Из особенностей: нет методов для чтения и записи файлов; конкретные хранилища подключаются через адаптеры; есть адаптер для FTP\cite{ftp}-сессии, некоторые методы не реализованы, что-то потенциально работающее написано только для платформы JVM; есть адаптер для локальной
                файловой системы (аналогично реализация есть только для JVM).
            }
            \existingsolution{files}{\cite{gh-asoft-files}}{
                Последнее изменение кода --- ноябрь 2020 г., в репозитории есть минимальный сопровождающий текст.
                Из кода ясно, что (возможно) поддерживаются платформы JVM, Android, JS (browser).
                Предоставляется мультиплатформенный интерфейс файла, но только с возможностью чтения. Судя по коду, данное решение позволяет запрашивать у пользователя загрузку файла в веб-браузере.
            }
            \existingsolution{kotlinx-fs}{\cite{gh-kotlinx-fs}}{
                Последнее изменение кода --- февраль 2019 г., в репозитории есть минимальный сопровождающий текст. Заявлена поддержка JVM, JS (Node.js) и POSIX-совместимых операционных систем (Mac OS X и Linux) для нативных Kotlin-приложений. Предоставляется интерфейс локальной файловой системы (проверка существования пути, чтение атрибутов путей, создание файлов и папок, в том числе временных, перемещение и удаление путей, чтение и запись файлов) и его реализации на перечисленных платформах.
            }
            \existingsolution{supernatural-fs}{\cite{gh-supernatural-fs}}{
                Последнее изменение кода --- декабрь 2020 г., есть минимальная документация. Предоставляется мультиплатформенный интерфейс локальной файловой системы, доступный на платформах Android, iOS, JS (Node.js), JVM.
            }
            \existingsolution{korio}{\cite{gh-korio}}{
                Последнее изменение --- август 2021 г., есть документация. Поддерживает очень много платформ: помимо Android, JVM, JS (browser), ещё и iOS, Mac OS X, Linux x64, watchOS, tvOS и другие. Библиотека является частью большого проекта (множества библиотек) Korlibs\cite{korlibs}, берущего начало в 2017 году и направленного на создание мультиплатформенного движка для видеоигр и сопутствующих мультиплатформенных библиотек. В связи с этим, данные библиотеки сильно завязаны друг на друга, и, вероятно, разрабатывались для написания приложений именно в среде этого набора библиотек. 
             
                Библиотека представляет свои реализации TCP- и HTTP-клиентов, примитивы для работы с потоковой обработкой данных (\code{AsyncStream}), собственную сериализацию/десериализацию форматов JSON, YAML, XML.

                Есть абстракция виртуальной файловой системы \code{Vfs}, элементы иерархии --- \code{VfsFile}, то есть нет различия на уровне типов между папками и файлами, но есть методы для определения этого (\code{isDirectory()}), поддерживаются атрибуты файлов, наблюдение за событиями файловой системы (\code{watch(path: String, handler: (FileEvent) -> Unit)}).\\ Есть методы для чтения и записи файлов как целиком, так и в потоковом режиме (используя \code{AsyncStream}).
                
                Среди доступных реализаций \code{Vfs} можно отметить \code{ZipVfs} --- виртуальную файловую систему (ВФС) для чтения zip-архивов, \code{UrlVfs} --- ВФС для чтения и записи ресурсов, расположенных в сети Интернет, по протоколу HTTP. Чтение доступно в потоковом режиме, запись --- только целиком; реализация предполагает, что данные можно записать, отправив PUT-запрос с содержимым на удаленный сервер. Есть также \code{LocalVfs} для работы с локальным файловым хранилищем с несколькими реализациями для разных платформ.
            }
            \existingsolution{okio}{\cite{gh-okio}}{
                Проект активно развивается, доступна документация.
                Является мультиплатформенным проектом, 
                \todo{описать FileSystem и возможности IO, sink, source}
            }
        \end{itemize}
    %subsubsection

    \subsubsection{Key-value решения}
        Здесь перечислены найденные продукты с открытым исходным кодом, которые представляют собой мультиплатформенные библиотеки, позволяющие персистентно сохранять пары ключ-значение в локальном хранилище. Они не решают поставленную задачу или ее части, однако в некоторых случаях их функциональности может оказаться достаточно, поэтому они стоят упоминания.

        Многие из этих проектов в качестве хранилищ используют\\ \code{SharedPreferences} для Android, \code{window.localStorage} для JS (browser), \code{NSUserDefaults} для iOS.

        \begin{itemize}
            \existingsolution{multiplatform-settings}{\cite{gh-multiplatform-settings}}{
                Проект активно развивается, доступна документация. Помимо платформ JVM, JS (browser) и Android поддержаны также iOS, macOS, watchOS, tvOS и Windows. Предоставляет интерфейс для сохранения пар ключ-значение в локальном хранилище.
            }
            \existingsolution{KVault}{\cite{gh-kvault}}{
                Проект активно развивается, есть минимальная документация. Поддержаны только платформы iOS и Android, однако реализована поддержка шифрования данных.
            }
            \existingsolution{Kissme}{\cite{gh-kissme}}{
                Последнее изменение --- январь 2020 г., есть минимальная документация. Аналогично предыдущему пункту, поддержаны платформы iOS и Android, реализовано шифрование данных.
            }
            \existingsolution{multiplatform-preferences}{\cite{gh-multiplatform-preferences}}{
                Проект заархивирован, последнее изменение --- февраль 2020 г., есть минимальная документация. Поддержаны платформы iOS и Android.
            }
            \existingsolution{Kotlin Data Storage}{\cite{gh-kds}}{
                Последнее изменение --- октябрь 2021 г., есть минимальная документация. 
                Поддержаны платформы JVM, JS (Node.js и browser), Android. Решение позволяет сохранять сериализуемые данные в локальных хранилищах (в т.ч. в файлы на платформе JVM). Благодаря использованию делегации свойств\cite{kotlinlang-delegated-properties}, работа с данными происходит в виде чтения и записи значений переменных.
            }
            \existingsolution{asoft-storage}{\cite{gh-asoft-storage}}{
                Последнее изменение кода --- апрель 2020 г., нет документации и примеров.  Из кода ясно, что (возможно) поддерживаются платформы JVM, Android, JS (browser). Не реализована логика работы на JVM. 
            }
            \existingsolution{kached}{\cite{gh-kached}}{
                Последнее изменение кода --- октябрь 2020 г., есть небольшой сопровождающий текст.
                Из кода ясно, что (возможно) поддерживаются платформы JVM, Android, iOS. 
                Для JVM реализовано хранилище на базе файловой системы.
                Из особенностей: поддерживается шифрование значений, хранение данных в Amazon S3\cite{amazon-s3} (но доступно только на платформе JVM).
            }
        \end{itemize}
    %subsubsection

    \subsubsection{Решения для мультиплатформенного ввода-вывода (IO)}
        Здесь перечислены решения, предоставляющие возможности для мультиплатформенного ввода-вывода данных. На их базе гипотетически можно построить решение главной задачи.
        \begin{itemize}
            \existingsolution{korio}{\cite{gh-korio}}{См. раздел \ref{existing-vfs-solutions}}.
            \existingsolution{okio}{\cite{gh-okio}}{
                См. раздел \ref{existing-vfs-solutions}
            }
            \existingsolution{kotlinx-io}{\cite{gh-kotlinx-io}}{
                Официальная библиотека для мультиплатформенного ввода-вывода. Последнее изменение кода --- май 2020 г., в вопросе об актуальности библиотеки\cite{gh-kotlinx-io-issue} был дан ответ, что в разработке новая версия этой библиотеки. По всей видимости, проект стал частью Ktor\cite{ktor} и развивается внутри него.
            }
            \existingsolution{tinlok}{\cite{gh-tinlok}}{
                \todo{}
            }
        \end{itemize}
    %subsubsection
    \subsubsection{Вывод}
        \todo{korio и okio предоставляют какие-то решения, но ни одно решение вообще не предоставило доступ к файловому хранилищу в браузере} 
    %subsubsection
%subsection

%section
