\section{Детали реализации} \label{solution-details}

\subsection{Структура проекта}
    \makefig{modules}{0.5\textwidth}{\label{fig:modules}
        Структура модулей проекта \code{multifs}.
    }
    
    Библиотека \code{multifs}\cite{gh-multifs} разделена на пять модулей, представленных на рисунке~\ref{fig:modules}. Модуль Common содержит определения базовых интерфейсов библиотеки, а также основную часть логики для поддержки Google Drive (\code{GoogleDriveAPI} и \code{GoogleDriveFS}). Модуль CommonJvmAndroid нужен для реализации поддержки \code{SystemFS}, так как необходимая для этого функциональность доступна на обеих платформах JVM и Android. Модули JVM и JS (browser) содержат реализации интерфейса \code{GoogleAuthorizationRequester}, обеспечивающих получение кода доступа к Google Drive пользователя. В модуле Android, помимо реализации этого интерфейса, содержится также реализация \code{SqliteFS}.
    
%subsection

\subsection{Сборка проекта}
    Сборка проекта осуществляется с помощью системы автоматической сборки Gradle\cite{gradle} версии 7.1, используемая версия JDK --- OpenJDK 11. Скрипт сборки написан в файле \code{build.gradle.kts} в корне проекта. Используемая версия языка Kotlin --- 1.6.20.

    Осуществить сборку и опубликовать артефакты в локальный maven-репозиторий можно с помощью команды \code{./gradlew publishToMavenLocal}.
%subsection

%\subsection{\code{SystemFS}}
%    \todo{реализует также \code{StreamingIO}}
%subsection

\subsection{\code{GoogleDriveFS}}
    \subsubsection{\code{GoogleDriveAPI}}
        \code{GoogleDriveAPI} использует HTTP-клиент библиотеки Ktor с подключаемым модулем \code{Auth}, который самостоятельно обеспечивает добавление HTTP-заголовка \code{Bearer} с ключом доступа и при необходимости вызывает методы для получения или обновления этого ключа.

        Так как большинство реализованных методов по своей механике не сильно друг от друга отличаются, здесь будет дан обзор наиболее интересных из них. 

        \paragraph{Метод list.} В качестве аргумента принимает идентификатор папки, список детей которой необходимо получить. 

        \todo{обзор реализованных методов}

        \todo{в том числе StreamingIO}
    %subsubsection
    \subsubsection{\code{GoogleAuthorizationRequester} и его реализации}
        \todo{Как на десктопе}

        \todo{как на андроиде}
        
        \todo{как в вебе}
    %subsubsection
%subsection

\subsection{\code{SqliteFS}}
    \todo{как устроено, что ему нужно}

    \todo{запросы}
%subsection

\subsection{Тестирование}
    \todo{тесты для systemfs}

    \todo{еще руками что-то потыкал через реализованные приложения (в т.ч. StreamingIO гугла)}
    
%subsection

% section
