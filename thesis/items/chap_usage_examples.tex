\section{Примеры использования библиотеки}\label{usage-examples}

\subsection{Сборка проекта}
    Сборка проекта осуществляется с помощью системы автоматической сборки Gradle\cite{gradle} версии 7.1, используемая версия JDK --- OpenJDK 11. Скрипт сборки написан в файле \code{build.gradle.kts} в корне проекта. Используемая версия языка Kotlin --- 1.6.20.

    Осуществить сборку и опубликовать артефакты в локальный maven-репозиторий можно с помощью команды \code{./gradlew publishToMavenLocal}.
%subsection

\subsection{Multieditor}
    \subsubsection{Подключение библиотеки}
        \todo{собираем либу, гредл команда, чтобы в локальный репозиторий артефакты попали}

        \todo{что дописать в гредл}
    % subsubsection
    \subsubsection{Реализация бизнес-логики в общем модуле приложения}
        \todo{ну тут понятно, что написано в \code{AppState.kt}}
    % subsubsection
    \subsubsection{Определение списка доступных хранилищ на каждой из целевых платформ}
        \todo{тут что написано на каждой платформе и как}
    % subsubsection
    \subsubsection{Реализация клиентских приложений на разных платформах}
        \todo{Общие слова про компоуз}
        \paragraph{\todo{Клиент для платформ JVM и Android.}}
        \paragraph{\todo{Клиент на платформе JS (browser).}}
    % subsubsection
%subsection

\subsection{gdrive-cli}
    \subsubsection{Написание логики приложения с учетом нескольких типовых ограничений на функциональность VFS}
        \todo{тут описать как писать логику, если нужна расширенная функциональность от вфс}

        \todo{как написан runCli}

        \todo{заодно как пользоваться StreamingIO}
    %subsection
%section
