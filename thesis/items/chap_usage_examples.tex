\section{Примеры использования библиотеки}\label{usage-examples}

\subsection{Сборка проекта}
  Сборка проекта осуществляется с помощью системы автоматической сборки Gradle\cite{gradle} версии 7.1, используемая версия JDK --- OpenJDK 11. Сценарий сборки написан в файле \code{build.gradle.kts} в корне проекта. Используемая версия языка Kotlin --- 1.6.20.

  Осуществить сборку и опубликовать артефакты в локальный maven-репозиторий можно с помощью команды \code{./gradlew publishToMavenLocal}.
%subsection

\subsection{Multieditor}
  Multieditor\cite{gh-multieditor} --- это мультиплатформенный текстовый редактор, доступный на платформах JVM, Android и Web и предоставляющий возможность использовать все три поддержанных ВФС: \code{SystemFS}, \code{GoogleDriveFS}, \code{SQLiteFS}.

  \subsubsection{Подключение библиотеки}
    Подключение библиотеки \code{multifs} осуществляется через сценарий автоматической сборки в файле \code{build.gradle.kts}. Поскольку на текущий момент библиотека ещё не опубликована в популярных maven-репозиториях, используется локальный maven-репозиторий. В блок \code{dependencies} общего модуля проекта добавлена зависимость
    \begin{minted}{kotlin}
implementation("dev.salavatov:multifs:${Versions.multifs}").
    \end{minted}
    В модули для платформ jvm, android и js добавлена соотвествующая зависимость на платформоспецифичный артефакт этой библиотеки: \code{multifs-jvm}, \code{multifs-android} или \code{mutlifs-js}.
  % subsubsection
  \subsubsection{Реализация бизнес-логики в общем модуле приложения}
    Вся основная логика приложения написана в файле \code{AppState.kt} в общем модуле.
    Класс \code{AppState} содержит поля, представляющие состояние области ввода текста (информация о текущем файле, содержимое, происходит ли сохранение данных в текущий момент), дерева навигации (список доступных для подключения хранилищ, состояние деревьев подключенных виртуальных файловых систем), а также последней ошибки.

    Этот класс также содержит собственную область видимости сопрограмм, диспетчеризацию которой осуществляет диспетчер сопрограмм, получаемый из функции \code{MultifsDispatcher()}~--- это функция, объявленная с помощью \code{expect/actual}-механизма. \code{actual}-объявление для платформ JVM и Android находится в общем для них модуле и возвращает \code{Dispatchers.IO}. Результатом на платформе JS (browser) является \code{Dispatchers.Unconfined}. 

    Определен метод \code{launchSafe}, который принимает в качестве аргумента произвольный \code{suspend}-блок кода и запускает его в упомянутой области видимости сопрограмм, при этом все исключения этого блока ловятся и сохраняются в поле \code{errorState} для возможности демонстрации ошибки пользователю.

    С помощью этого метода уже реализуется непосредственно бизнес-логика приложения (10 функций), несколько из них для примера: 

    \begin{minted}[fontsize=\small]{kotlin}
fun launchFileOpen(targetFile: File) =
  launchSafe {
    val contentBytes = targetFile.read()
    if (!editor.saving) {
      editor.content = contentBytes.decodeToString()
      editor.file = targetFile
    }
  }
    \end{minted}

    \begin{minted}[fontsize=\small]{kotlin}
fun launchInitializeStorage(namedStorageFactory: NamedStorageFactory) =
  launchSafe {
    val fs = namedStorageFactory.init()
    val name = namedStorageFactory.name
    navigation.availableStoragesState.remove(namedStorageFactory)
    navigation.configuredStoragesState.add(
      FileTree(
        fs,
        name,
        FolderNode(fs.root, null, mutableStateListOf())
     )
    )
  }
    \end{minted}

    В последней приведенной функции конструктор \code{FolderNode} имеет параметры: \code{Folder} из \code{multifs}, ссылка на родительский \code{FolderNode}, состояние списка дочерних узлов (оно обновляется при раскрытии/скрытии подробной информации этой папки).

    Реализации графических интерфейсов для изменения состояния приложения вызывают только методы класса \code{AppState}.

  % subsubsection
  \subsubsection{Определение списка доступных хранилищ на каждой из целевых платформ}
    Как можно было заметить в коде выше, бизнес-логика приложения работает со списком доступных хранилищ \code{navigation.availableStoragesState}. Для работы с хранилищами  в общем коде создан вспомогательный класс и синоним типа:
    \begin{minted}[fontsize=\small]{kotlin}
typealias Storage = VFS<out File, out Folder>
class NamedStorageFactory(val name: String, val init: suspend () -> Storage)
    \end{minted}

    Получение списка доступных хранилищ на каждой платформе вынесено в функцию или класс:
    \begin{minted}[fontsize=\small]{kotlin}
// JVM
fun makeStorageList(): List<NamedStorageFactory> {
  val systemfs = NamedStorageFactory("Local device") { SystemFS() }
  val gdfs = NamedStorageFactory("Google Drive") {
    val googleAuth = CacheGoogleAuthorizationRequester(
      HttpCallbackGoogleAuthorizationRequester(
        GoogleAppCredentials(CLIENT_ID, SECRET),
        GoogleDriveAPI.Companion.DriveScope.General
      )
    )
    val gapi = GoogleDriveAPI(googleAuth)
    GoogleDriveFS(gapi).also {
      it.root.listFolder() // trigger auth
    }
  }
  return listOf(systemfs, gdfs)
}
    \end{minted}
    Вызов \code{listFolder} на корне ВФС Google Drive нужно, чтобы при инициализации хранилища сразу произошел запрос авторизации.
    \begin{minted}[fontsize=\small]{kotlin}
// JS
fun makeStorageList(): List<NamedStorageFactory> {
  val gdfs = NamedStorageFactory("Google.Drive") {
    val googleAuth =
      PopupGoogleAuthorizationRequester(
        GoogleDriveAPI.Companion.DriveScope.General
      )
    val gapi = GoogleDriveAPI(googleAuth)
    GoogleDriveFS(gapi).also {
      it.root.listFolder() // trigger auth
    }
  }
  return listOf(gdfs)
}
    \end{minted}
    \begin{minted}[fontsize=\small]{kotlin}
// Android
class Storages(
  private val googleAuth: IntentGoogleAuthorizationRequester,
  private val sqliteFSDbHelper: SqliteFSDatabaseHelper,
  private val systemFSRoot: Path
) {
  private val sqlite = NamedStorageFactory("SQLite") {
      SqliteFS(sqliteFSDbHelper)
  }

  private val googleDrive = NamedStorageFactory("Google Drive") {
    val gapi = GoogleDriveAPI(googleAuth)
    GoogleDriveFS(gapi).also {
        it.root.listFolder() // trigger auth
    }
  }

  private val systemFS = NamedStorageFactory("Local") {
    SystemFS(systemFSRoot)
  }

  fun asList(): List<NamedStorageFactory> {
    return listOf(googleDrive, sqlite, systemFS)
  }
}
    \end{minted}
    В случае Android это реализовано в виде класса, который в конструкторе принимает необходимые для инициализации ВФС компоненты. Первый из них~--- это экземпляр \code{IntentGoogleAuthorizationRequester}. Как было упомянуто в разделе~\ref{solution-details-gdrive-auth-req}, для создания экземпляра этого класса необходимо определить лямбда-функцию, которая запускает переданный в нее \code{Intent} и возвращает результат этого намерения. Клиент данного приложения для платформ JVM и Android написан с помощью мультиплатформенной библиотеки Compose\cite{gh-compose-jb}. С помощью средств этой библиотеки возможно реализовать создание экземпляра класса \code{IntentGoogleAuthorizationRequester}, не изменяя существующий код:
    \begin{minted}[fontsize=\small]{kotlin}
@Composable
fun makeGoogleAuthorizationRequester(
  googleAppCredentials: GoogleAppCredentials,
  scope: GoogleDriveAPI.Companion.DriveScope
): IntentGoogleAuthorizationRequester {
  var resultFuture = remember { CompletableDeferred<ActivityResult>() }
  val startForResult =
    rememberLauncherForActivityResult(
        ActivityResultContracts.StartActivityForResult()
    ) {
        resultFuture.complete(it)
    }
  return IntentGoogleAuthorizationRequester(
    googleAppCredentials,
    scope,
    LocalContext.current,
  ) { intent ->
    resultFuture = CompletableDeferred<ActivityResult>()
    startForResult.launch(intent)
    resultFuture.await()
  }
}
    \end{minted}
    Как видно из листинга, используется аннотация \code{@Composable}, а также функция \code{rememberLauncherForActivityResult}. 
    
    Можно обойтись и стандартными средствами библиотеки Android, но возникнет необходимость переопределить метод \code{onActivityResult} класса \code{Activity} для обработки результатов запущенного намерения, либо зарегистрировать \code{ActivityResultLauncher} в существующей \code{Activity} с помощью \code{registerForActivityResult}.
    
  % subsubsection
  \subsubsection{Реализация клиентских приложений на разных платформах}
    Для платформ JVM и Android код клиента с графическим интерфейсом пользователя написан в общем для них модуле с помощью библиотеки Compose\cite{gh-compose-jb}. Для платформы JS (browser) клиент реализован аналогично с помощью библиотеки Compose for Web.

    \makefig[H]{multieditor-jvm-start-small}{\textwidth}{\label{fig:multieditor-jvm-start}
      Multieditor JVM сразу после запуска.
    }

    При запуске приложения пользователю в меню навигации доступны несколько ВФС, которые можно подключить (рисунок~\ref{fig:multieditor-jvm-start}). При попытке подключить Google Drive открывается вкладка в браузере, где пользователю предлагается выбрать учетную запись Google (рисунок~\ref{fig:google-auth}). После выбора аккаунта можно пользоваться подключенной ВФС как обычной файловой системой (рисунок~\ref{fig:multieditor-jvm}). Пользователю доступен для использования весь функционал библиотеки (создание и удаление папок и файлов, просмотр файлов, их сохранение, перемещение, копирование).

    \makefig[H]{multieditor-jvm-signin}{0.47\textwidth}{\label{fig:google-auth}
      Процесс подтверждения авторизации на сервисе Google, снимок экрана вкладки в веб-браузере (Multieditor JVM).
    }
    \makefig[H]{multieditor-jvm}{\textwidth}{\label{fig:multieditor-jvm}
      Multieditor JVM в процессе использования с подключенным хранилищем Google Drive.
    }
    
    Тот же самый клиент доступен на платформе Android  (рисунки~\ref{fig:multieditor-android-signin}, \ref{fig:multieditor-android}, \ref{fig:multieditor-android-create-file}). Для платформы JS (browser) клиент выглядит слегка иначе, но предоставляет в точности те же самые возможности (рисунок~\ref{fig:multieditor-web}).

    \makefig[H]{multieditor-android-signin}{\textwidth}{\label{fig:multieditor-android-signin}
      Процесс подтверждения авторизации на сервисе Google (Multieditor Android).
    }
    \makefig[H]{multieditor-android}{\textwidth}{\label{fig:multieditor-android}
      Снимок экрана Multieditor Android с подключенными хранилищами Google Drive и SQLite.
    }

    Для создания диалоговых окон библиотека Compose предоставляет \code{AlertDialog}, однако на текущий момент при его использовании возникают различные ошибки\cite{gh-compose-alert-dialog,gh-compose-alert-dialog-2}, поэтому используется собственная реализация \code{AlertDialog} через \code{expect/actual}-механизм: на платформе JVM используется \code{Dialog} для создания нового системного окна с переопределенными декорациями, на Android~--- тоже \code{Dialog}, но ввиду специфики код несколько отличается. Для JS (browser) в код страницы добавляется глобальный элемент модального окна, который управляется через модификацию содержимого и CSS-стили.

    \makefig[H]{multieditor-android-create-file}{\textwidth}{\label{fig:multieditor-android-create-file}
      Всплывающее диаологовое окно Multieditor Android для добавления файла или папки.
    }
    \makefig[H]{multieditor-web}{\textwidth}{\label{fig:multieditor-web}
      Снимок экрана Multieditor Web в веб-браузере.
    }
  % subsubsection
%subsection

\subsection{gdrive-cli}
  gdrive-cli\cite{gh-gdrive-cli} --- это приложение, предоставляющее возможность работать с хранилищем Google Drive через интерфейс командной строки. Позволяет в том числе загружать и скачивать файлы в потоковом режиме. Писалось с целью проверки работоспособности данной функциональности.

  \subsubsection{Написание логики приложения с учетом нескольких типовых ограничений на функциональность VFS}
    Логика приложения написана в функции \code{runCli}, имеющей следующую сигнатуру:
    \begin{minted}[fontsize=\small]{kotlin}
suspend fun <T> runCli(fs: VFS<out T, out Folder>)
        where T : File, T : StreamingIO {
  ...
}
    \end{minted}
    Как видно по типу ВФС, данная функция может принимать в качестве \code{fs} не только \code{GoogleDriveFS}, но также любую другую виртуальную файловую систему, класс файла которой реализует расширение \code{StreamingIO}.

    Для примера разберем как реализована команда скачивания файла:
    \begin{minted}[fontsize=\small]{kotlin}
when (command) {      
  // ...
  is Download -> {
    val node = list[command.index - 1] // Узел, который хотим скачать
    // Проверяем, что мы хотим скачать файл, а не папку. Поскольку тип
    // node - это VFSNode, необходимо приведение типов также к StreamingIO.
    // Можно обойтись без явной проверки, поскольку файл класса точно 
    // реализует это расширение. В данном случае с помощью этой проверки 
    // срабатывает умное приведение типов Kotlin.
    if (node !is File || node !is StreamingIO) 
      throw Exception("cannot download a directory")
    // Получаем канал, из которого можно читать содержимое файла
    val dlStream = node.readStream()
    // Определяем в какой файл записывать данные
    val destPath = Paths.get(command.destPath)
    if (destPath.exists()) {
      destPath.deleteExisting()
    }
    val destFile = destPath.toFile()
    while (!dlStream.isClosedForRead) {
      // Пока в канале есть данные, считываем их по частям и дополняем
      // файл, попутно логируем в консоль количество записанных байт
      dlStream.read { buffer ->
        print("\rbytes: ${dlStream.totalBytesRead}")
        destFile.appendBytes(buffer.moveToByteArray())
      }
    }
    // Пишем в консоль размер скачанного файла
    println("\rtotal bytes downloaded: ${dlStream.totalBytesRead}")
  }
  // ...
}
    \end{minted}

    При использовании данной команды можно наблюдать, что в консоли в течение всего процесса скачивания обновляется сообщение о размере записанных байт. 
    
    Схожим образом написана команда загрузки данных в Google Drive. 

    С помощью данной утилиты проверена функциональность потокового чтения и записи файлов для ВФС \code{GoogleDriveFS}: были несколько раз успешно загружены и скачаны файлы размером в 50 МиБ.

  %subsection
%section
