\specialsection{Постановка задачи}

Цель работы состоит в разработке мультиплатформенной библиотеки на языке программирования Kotlin, позволяющей разработчику описывать логику работы с различными файловыми хранилищами в общем модуле мультиплатформенного проекта. Библиотека должна предоставлять интерфейс виртуальной файловой системы с иерархической структурой папок и файлов и позволять записывать и читать файлы как массивы байт. Библиотека должна быть достаточно гибкой, чтобы разработчик мог самостоятельно дополнить её функциональность (например, поддержать на базе библиотеки новое облачное хранилище), а также иметь возможность расширения на уровне предоставляемых интерфейсов, когда от целевых хранилищ требуется поддержка особых возможностей (например, наличие у файлов атрибутов прав на чтение/запись). 

От конечного продукта ожидается как минимум:
\begin{itemize}
    \item поддержка трех платформ: JVM (для приложений, работающих в среде операционных систем Windows, Linux, и т.п.), JS (браузерные приложения), Android (мобильные приложения);
    \item поддержка как минимум одного облачного хранилища, доступного со всех поддерживаемых платформ.
\end{itemize}

% Предлагается следующий план выполнения задачи:
% \begin{enumerate}
%     \item разобраться в предметной области, выделить ключевые проблемы;
%     \item провести исследование существующих решений с открытым исходным кодом;
%     \item с учетом полученных данных разработать архитектуру библиотеки;
%     \item реализовать библиотеку, а также поддержку нескольких хранилищ для примера;
%     \item написать относительно простое мультиплатформенное приложение на базе библиотеки, иллюстирующее преимущества и недостатки получившегося решения;
%     \item провести анализ получившегося продукта.
% \end{enumerate}

