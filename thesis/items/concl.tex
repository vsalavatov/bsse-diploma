\specialsection{Заключение}

%\todo{Заключение должно подводить итоги работы и содержать информацию о полученных в рамках работы результатах.}
%\todo{написана либа, поставленная задача выполнена, на ее базе написаны примеры}
%\todo{есть куда расти в плане поддержки платформ, бекендов, фичей вфс}

Проведено исследование существующих решений и выявлены их достоинства и недостатки.

Основным результатом данной работы является реализованная мультиплатформенная Kotlin-библиотека \code{multifs}, позволяющая описывать логику работы с виртуальной файловой системой в общем коде. Данная библиотека предоставляет поддержку трех файловых хранилищ, среди которых одно облачное, доступное на платформах JVM, Android и JS (browser). Наличие поддержки браузерной платформы является существенным преимуществом разработанного решения.

На базе созданной библиотеки реализовано два приложения, одно из которых мультиплатформенное с клиентами на трёх перечисленных платформах и поддерживающее все три файловых хранилища. Данное приложение демонстрирует возможности и процесс использования \code{multifs} с точки зрения разработчика.
